
% Abstract File
% Do not remove centering environment below
\begin{center}

\end{center}

\begin{center}
{{\bf\fontsize{14pt}{14.5pt}\selectfont \uppercase{ABSTRACT}}}
\end{center}

\doublespacing
\addcontentsline{toc}{chapter}{Abstract}



\begin{center}
	\begin{doublespace}
%{\fontsize{14pt}{14.5pt}\selectfont {Applications of Cubic Spline Interpolation On Functions in Normed Spaces}}\\
		%{\fontsize{14pt}{14.5pt}\selectfont {Cory Suzuki}}\\
%		{\fontsize{12pt}{12.5pt}\selectfont {Ph.D. Year}}\\
	\end{doublespace}
\end{center}


In this paper, we conduct a detailed error analysis of two cubic spline interpolation methods described in Kreyszig and Mhaskar-Pai in the $L^2$ and $L^{\infty}$ norms. We begin by introducing the basic theory of cubic spline interpolation and dissecting each of the interpolation conditions from both methods. The data was collected by constructing splines of various elementary functions and calculating the relative $L^2$ and $L^{\infty}$ error between each function and its spline. The numerical data has shown that the functions $x^4$, $\cos(x)$, and $\ln(x+1)$ using both methods of interpolation approximate their functions with minimal relative error. We show that polynomial-trigonometric functions are the worst in the $L^2$ and $L^{\infty}$ norms. We conjecture that higher-order powers of polynomial-trigonometric functions such as $x^n\sin(x)$ will always perform worse than the polynomial function $x^{n}$ in the $L^2$ and $L^{\infty}$ norms. In addition, we compared the actual Mhaskar-Pai error to the error bounds given in \cite[pg.~269]{key5}. We observe how sharp the bound was for various elementary functions.
