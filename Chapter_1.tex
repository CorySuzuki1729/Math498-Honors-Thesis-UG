
% Chapter 2 File

\chapter{Introduction}
\label{chapter1}
\thispagestyle{empty}

 Approximation theory is a field of mathematics that studies special algorithms that aid mathematicians and scientists in making approximations of objects like functions and data. Functions in particular are quite interesting as they represent real-life phenomena as abstractions, which in turn are then used to predict other natural phenomena. Without approximation theory, we would not have many important inventions such as spaceships, airplanes, and computers which we often take advantage of.
\\\\
An \emph{algorithm} is a well-defined list of steps that are used to produce an effective solution to a specific problem instance. Algorithms are malleable and can be modified to solve other similar problem instances. For example, consider the Newton-Raphson and the Secant algorithms, which originate from the field of numerical analysis. These algorithms are used to approximate the roots of functions on specific domains with appropriately defined tolerance values. They both achieve this goal, so we say that these algorithms belong to a \emph{class} of problem instances, in this case, root approximation. However, it is important to note that these algorithms do not solve the same problem instance. The Newton-Raphson algorithm requires that the derivative of a function be known, unlike the Secant algorithm which utilizes the secant variation of the derivative definition in its process \cite{key3}. Knowing what an algorithm is, we now introduce the motivation for researching cubic spline interpolation. We will be interested in studying two different algorithms for constructing cubic splines and analyzing the approximation of elementary functions in the $L^2$ and $L^{\infty}$ norms.


\section{Motivation for Research: Why Do We Study Cubic Splines?}
The fields of approximation theory and functional analysis are fundamentally crucial in understanding how accurate the approximations of functions are. With these tools in our arsenal, we have the ability to provide accurate estimates of arbitrary functions and compute the errors in such calculations. The type of approximation we primarily study is \emph{cubic spline interpolation}. This method of approximation has been proven useful in recent developments such as filling in missing portions of weather data using the statistical method of time series \cite{key1}. Another significant motivation for studying splines is that basis spline interpolation is utilized for image processing and digital filtering \cite{key2}.
\\\\
We want to study how we can use certain algorithms from approximation theory in order to make accurate approximations of both known and unknown functions. We denote \emph{known functions} to be functions that you may recall from high school such as polynomials and logarithms, while \emph{unknown functions} are functions that may be a combination of elementary known functions, or may even be defined piece-wise based on data. This leads us to the heart of the research on the effectiveness of cubic spline interpolation. A cubic spline can be utilized to our advantage by producing approximations of functions and observing how well they estimate said functions. For instance, we are interested in finding the approximation error and comparing that to the upper bounds established in the literature. Another factor we will consider is how changing the number of data points in a partition effects the approximation. These questions will guide our research on how cubic spline interpolation can be optimized under hypothetical constraints. 