
% Conclusion File

\chapter{Concluding Remarks and Future Work}
\thispagestyle{empty}
\section{Summary}

In this thesis, we introduced the classical theory of cubic spline interpolation and focused on two types of cubic spline interpolation from Kreyszig and from Mhaskar-Pai. After constructing the cubic splines for each model using Mathematica code, data analysis was performed based on the two different measures of relative error, as well as Mhaskar-Pai error-bound calculations. For both the Kreyszig and Mhaskar-Pai splines we concluded that the cubic spline approximations for both models of the functions $x^4$, $\ln(x+1)$, and $\cos(x)$ performed the best with the minimum amount of $L^2$ and $L^{\infty}$ relative error. The functions $x^4$, $\ln(x+1)$, and $\cos(x)$ were closest to the maximum error predicted by Mhaskar-Pai.
\section{Future Work}
We conclude this paper by providing some insight into how we can approach more immediate future work and present some prospective ideas for distant future work. Due to the limited time we had, it would be beneficial to continue running more mixtures of elementary functions with the same partitions and intervals that we presented in the data tables. For example, we would like to further test the conjecture that $x^n\sin(x)$ has a worse-performing spline approximation than $x^n$ for larger values of $n$. In addition, we can also consider testing the functions we had with a greater amount of unbalanced partitions, and also include different intervals such as $[-4,4]$. These are merely a few possibilities that can be studied in order to develop more conjectures about the algorithms used in cubic spline interpolation.\\\\
It is believed that this research will provide the theoretical groundwork for deeper research to be implemented on the type of function class we investigated. As mentioned previously, cubic spline interpolation is a useful technique that allows scientists to make predictions about data sets. Future work could also consider the same analysis that we conducted in this thesis to cubic splines of multivariable functions in normed spaces. Our hope is that the next curious mind who stumbles upon this paper within the university community will be enlightened in the way mathematics research is conducted and can have a starting place in learning about the power of mathematics and its numerous real-world applications.